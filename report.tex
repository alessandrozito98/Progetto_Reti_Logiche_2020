%%%%%%%%%%%%%%%%%%%%%%%%%%%%%%%%%%%%%%%%
% University/School Laboratory Report
% LaTeX Template
% Version 3.1 (25/3/14)
%
% This template has been downloaded from:
% http://www.LaTeXTemplates.com
%
% Original author:
% Linux and Unix Users Group at Virginia Tech Wiki 
% (https://vtluug.org/wiki/Example_LaTeX_chem_lab_report)
%
% License:
% CC BY-NC-SA 3.0 (http://creativecommons.org/licenses/by-nc-sa/3.0/)
%
%%%%%%%%%%%%%%%%%%%%%%%%%%%%%%%%%%%%%%%%%

%----------------------------------------------------------------------------------------
%	PACKAGES AND DOCUMENT CONFIGURATIONS
%----------------------------------------------------------------------------------------

\documentclass{article}
\usepackage[utf8]{inputenc}
\usepackage[italian]{babel}
\usepackage{appendix}
\usepackage{geometry}
\usepackage[T1]{fontenc}
\usepackage{siunitx} % Provides the \SI{}{} and \si{} command for typesetting SI units
\usepackage{graphicx} % Required for the inclusion of images
\usepackage{natbib} % Required to change bibliography style to APA
\usepackage{amsmath} % Required for some math elements 
\usepackage{caption}
\usepackage{tikz}
\usepackage{pdfpages}

\usetikzlibrary{arrows,automata, positioning}

\usepackage{import}

\setlength\parindent{0pt} % Removes all indentation from paragraphs

%\renewcommand{\labelenumi}{\alph{enumi}.} % Make numbering in the enumerate environment by letter rather than number (e.g. section 6)

%\usepackage{times} % Uncomment to use the Times New Roman font

%----------------------------------------------------------------------------------------
%	DOCUMENT INFORMATION
%----------------------------------------------------------------------------------------

\title{Prova Finale di Reti Logiche} % Title
\author{Giuseppe Serra \\ Alessandro Zito} % Author name
\date{A. A. 2020-21}

\begin{document}
\maketitle % Insert the title, author and date
\begin{center}
\begin{tabular}{l r}
Matricole: & 887630 - 890219\\ % Partner names
Codici Persona: & 10566090 - 10617579\\
Docente: & Gianluca Palermo % Instructor/supervisor
\end{tabular}
\end{center}

\newpage

\tableofcontents

\newpage


% If you wish to include an abstract, uncomment the lines below
% \begin{abstract}
% Abstract text
% \end{abstract}

%----------------------------------------------------------------------------------------
%	SECTION 1
%----------------------------------------------------------------------------------------




\section{Requisiti del Progetto}


La specifica della Prova finale (Progetto di Reti Logiche) 2020 è ispirata al metodo di \textbf{equalizzazione dell’istogramma di una immagine.}
Il metodo di equalizzazione dell’istogramma di una immagine è un metodo pensato per ricalibrare il contrasto di una immagine quando l ’intervallo dei valori di intensità sono molto vicini effettuandone una distribuzione su tutto l’intervallo di intensità, al fine di incrementare il contrasto. Al componente viene richiesto di:
\begin{enumerate}
\item Accedere ai primi due indirizzi della RAM per leggere i byte dell'immagine (indirizzo $0$ per le colonne e $1$ per le righe). La dimensione massima di un'immagine è 128x128 pixel.
\item Leggere ogni pixel dell'immagine
\item Calcolare il nuovo valore del pixel.
\item Scrivere il risultato nella memoria RAM, e rifare lo stesso processo dal punto 4 per tutti i pixel dell'immagine.
\end{enumerate}

Inoltre, l'implementazione deve essere in grado di gestire un segnale di Reset. Per l'implementazione si è scelto di supporre il Reset asincrono rispetto al segnale di clock.

L'implementazione deve essere poi sintetizzata con target FPGA xc7a200tfbg484-1.

\subsection{Esempio}



\subsection{Ipotesi Progettuali}
\label{ipotesi}
Si sono supposti veri i seguenti fatti:
\begin{enumerate}
\item Il nuovo valore del pixel di un massimo di 8 bit.
\item L'immagine non può essere di 0 pixel.
\item Il programma è sintetizzato in modo da poter codificare più immagini mantenendo i vincoli di RESET imposti dalla specifica. 
\end{enumerate}
 
%----------------------------------------------------------------------------------------
%	SECTION 2
%----------------------------------------------------------------------------------------

\section{Implementazione}

L'architettura è stata progettata in maniera modulare, in modo da specializzare i singoli componenti creati e separare le funzionalità
di calcolo della codifica dell'indirizzo dalla gestione della macchina a stati finiti.


\subsection{Descrizione ad Alto Livello}
\label{alto_livello}

Da un'ottica di alto livello, l'implementazione esegue i seguenti passi:

\begin{enumerate}
\item Legge il primo indirizzo e salva il numero delle colonne delle immagini
\item Legge il secondo indirizzo e salva il numero delle righe delle immagini
\item Calcola il numero dei byte dell'immagine
\item Legge ogni pixel dell'immagine e controlla se è il pixel con valore più grande o più piccolo dell'immagine:
    \begin{enumerate}
        \item Se sì, salva il valore in \textit{max\_pixel\_value} o \textit{min\_pixel\_value}
        \item Se no, legge il pixel successivo.
    \end{enumerate}
\item Calcola la differenza tra il pixel con valore maggiore e pixel con valore minore (\textit{delta\_value}).
\item Calcola il numero di bit da shiftare con la seguente formula:

    \begin{equation}
        shit\_level = Floor(\log_2(delta\_value + 1)
    \end{equation}
    
    Il seguente calcolo è stato effettuato tramite un controllo a soglia iterando sulla potenza di 2.

\item Si prende il valore del pixel e lo si shifta di shift\_level e codifica il numero a 16 bit aggiungendo "$00000000$" in coda:
\begin{enumerate}
    \item Se il numero è minore di 255, Carica il risultato nella RAM codificato a 8 bit;
    \item Altrimenti, carica nella RAM "$11111111$".
\end{enumerate}


\item Ripete i passaggi dal 7 in poi per tutti i pixel da leggere.
\item Si aggiornano tutti i bit di comando.
\end{enumerate}

Per gestire questo algoritmo si è scelta un'implementazione costituita da una macchina a stati finiti, che rappresenta il top-level component
e che gestisce tutto il processo di equalizzazione dell'immagine.


\subsection{Macchina a Stati Finiti}
\label{FSM}

La \textbf{Macchina a Stati Finiti} (FSM) è stata realizzata con specifica \textit{Behavioural} 
Segue una descrizione degli stati formali della FSM:

\begin{itemize}
\item \texttt{RESET}: Stato di idle in cui si posiziona la FSM al reset della computazione. La macchina aspetta che venga asserito il segnale \texttt{i\_start}.
\item \texttt{READ\_COLUMN}: Stato in cui viene letta la colonna dell'immagine
\item \texttt{READ\_ROW}: Stato in cui viene letta la riga dell'immagine.
\item \texttt{NUMBER\_OF\_BYTE}: Stato in cui si calcolano il numero di byte dell'immagine
\item \texttt{CALC\_MAX\_AND\_MIN}: Stato in cui vengono letti tutti i pixel e vengono trovati il pixel più grande e più piccolo
\item \texttt{READ\_PIXEL}: Stato in cui si aspetta un ciclo di clock per leggere un pixel e ritornare nello stato di calcolo di massimo e minimo 
\item \texttt{CALC\_SHIFT\_LEVEL}: Stato in cui si calcola lo \textit{shift\_level}
\item \texttt{READ\_PIXEL\_VALUE}: Stato in cui, dopo aver calcolato lo shift\_level, si ricominciano a leggere i valori dei pixel dell'immagine; se il contatore ausiliare è uguale al numero di byte, si va allo stato finale, altrimenti, si calcola il valore del pixel shiftato.
\item \texttt{SET\_NEW\_PIXEL\_VALUE}: Stato in cui viene calcolato il nuovo valore del pixel e codificato in 16 bit.
\item \texttt{SHIFT\_NEW\_PIXEL\_VALUE}: Stato in cui viene shiftato il nuovo valore del pixel, viene settato \textbf{o\_we} a $1$ per scrivere.
\item \texttt{WRITE\_NEW\_PIXEL\_VALUE}: Stato in cui viene scritto il nuovo valore del pixel sulla RAM sul suo indirizzo \textbf{o\_address} di uscita che vale esattamente \textit{o\_address = count + 2 + number\_of\_byte}.
\item \texttt{WAIT\_READ}: Stato in cui si aspetta un ciclo di clock per settare l'indirizzo di lettura uguale a count + 3 e settare \textbf{o\_we = 0}, per tornare a READ\_PIXEL\_VALUE
\item \texttt{FINALIZE}: Stato in cui si aspetta che \texttt{i\_start} venga portato giù, settando \texttt{o\_done} a 1 e riportandosi allo stato di \texttt{RESET}.
\end{itemize}

Più precisamente l'insieme degli stati della FSM dovrebbe comprendere almeno anche il counter \texttt{count}, che viene usato prima come contatore per leggere tutti gli stati di lettura dei pixel, poi viene utilizzato per il calcolo del logaritmo in base 2 per lo shift\_level e, infine, per gli indirizzi di lettura e scrittura sulla RAM.


Nella figura a seguito è riportato il disegno dell'automa. Sono state usate le seguenti abbreviazioni degli stati per chiarezza:\\

\begin{center}
\begin{tabular}{ll}
\texttt{RESET}: & \texttt{RST}\\
\texttt{READ\_COLUMN}: & \texttt{RD\_RC} \\
\texttt{READ\_ROW}: & \texttt{RD\_RW} \\
\texttt{NUMBER\_OF\_BYTE}: & \texttt{NUM\_OF\_B} \\
\texttt{CALC\_MAX\_AND\_MIN} (With \texttt{WZ\_COUNT=i}): & \texttt{MAX\_E\_MIN} \\
\texttt{READ\_PIXEL} (With \texttt{COUNT=i}): & \texttt{READ\_PX} \\
\texttt{CALC\_SHIFT\_LEVEL}: & \texttt{SHIFT} \\
\texttt{READ\_PIXEL\_VALUE}: & \texttt{READ\_PX\_VL} \\
\texttt{CALC\_NEW\_PIXEL\_VALUE}: & \texttt{CL\_PX\_VL} \\
\texttt{SET\_NEW\_PIXEL\_VALUE}: & \texttt{SET\_NEW\_PX} \\
\texttt{WRITE\_NEW\_PIXEL\_VALUE}: & \texttt{WRITE\_NEW\_PX} \\
\texttt{WAIT\_READ}: & \texttt{WAIT\_RD} \\
\texttt{FINALIZE}: & \texttt{END} \\
\end{tabular}
\end{center}


\begin{figure}
\centering
\resizebox{!}{0.95\textheight}{%
\begin{tikzpicture}[->,>=stealth',shorten >=1pt,auto,node distance=2.5cm,
                    semithick, initial where=above]

  \tikzstyle{every state} = [font=\small, inner sep = 1pt, minimum size= 50pt]
  \node[state] 		   	(A)                    	{RST};
  \node[state] 			(B) [below of = A] 	  	{\texttt{RD\_RC}};
  \node[state]         	(C) [below of = B] 	  	{\texttt{RD\_RW}};
  \node[state]         	(D) [below of = C] 	  	{\texttt{NUM\_OF\_B}};
  \node[state]         	(E) [below of = D]      {\texttt{C\_MAX\_MIN}};
  \node[state]         	(F) [left = 2cm of E]  	{\texttt{READ\_PX}};
  \node[state]         	(G) [below of = E]      {\texttt{SHIFT}};
  \node[state]         	(H) [below of = G]      {\texttt{READ\_NEW\_PX}};
  \node[state]         	(I) [below of = H]     	{\texttt{SET\_NEW\_PX}};
  \node[state]         	(J) [right = 4cm of H] 	{\texttt{WR\_NEW\_PX}};
  \node[state]         	(K) [right = 4cm of D]	{\texttt{END}};


  \path (A) edge              		node [pos=0.5]{\texttt{I\_START=1}} (B)
  		(A) edge 	[loop left] 	node {} (A)
        (B) edge              		node {} (C)
        (C) edge              		node {} (D)
        (D) edge 			  		node {} (E)
        (E) edge 	[bend left=45] 	node {} (F)
        (F) edge	[bend left=45]	node {} (E)
        (E) edge			  		node {} (G)
        (G) edge			  		node {} (H)
        (H) edge			  		node {} (I)
        (I) edge	[bend right]	node {} (J)
        (J) edge				  	node {} (H)
        (J) edge			  		node {} (K)
        (K) edge	[bend right]	node [pos=0.2]{\texttt{I\_START=0}} (A);
        
        
\end{tikzpicture}
}
\end{figure}
\captionof{figure}{Diagramma degli stati della Macchina a Stati Finiti}

\newpage



\subsection{Schema dell'Implementazione}
\label{schema}

\vspace*{1cm}

% Your image goes here

\begin{figure}[h!]
    \centering
    \includegraphics[scale=0.4]{schematic.png}
    \caption{Schema dell'implementazione}
    \label{fig:schematic.png}
\end{figure}



%----------------------------------------------------------------------------------------
%	SECTION 3
%----------------------------------------------------------------------------------------

\section{Test Benches}
\label{test}

I test effettuati hanno cercato di effettuare transizioni critiche oppure di verificare possibili configurazioni di memoria estreme. In particolare sono stati creati dei testbench
comprendenti le seguenti situazioni:

\begin{itemize}
\item Vari test generati casualmente tramite un generatore di immagini in python sia quadrate che rettangolari.
\item Caso specifico in cui tutti i pixel sono dello stesso valore
\item Caso specifico con \textit{min\_pixel\_value} = 0 e \textit{max\_pixel\_value} = 255
\end{itemize}

Questi test hanno evidenziato alcune criticità nel codice iniziale e hanno guidato lo sviluppo della soluzione fino alla sua versione finale.\\

Per tutti i test riportati e i test successivi è stata effettuata la simulazione behavioural e successivamente la simulazione functional e timing post-synthesis, tutte con successo.\\

I risultati rilevanti sono riportati nella sezione \ref{risultati}.

\section{Risultati Sperimentali}
\label{risultati}

\subsection{Report di Sintesi}

Dal punto di vista dell'area la sintesi riporta il seguente utilizzo dei componenti:
\begin{itemize}
\item LUT: 257 (0.63\% del totale)
\item FF: 127 (0.15\% del totale)
\end{itemize}
Si è fatta particolare cautela nella scrittura del codice per evitare utilizzo di Latch.

\subsection{Risultato dei Test Bench}

Lavorando sul testing si è anche provato a variare il periodo di clock, confermando che l'implementazione rispetta le specifiche, funzionando a \SI{100}{\ns}.\\

Si riportano i risultati di tempo legati agli ultimi 2 test riportati nella sezione \ref{test}\\

\begin{tabular}{ll}
Simulazione 1 immagine 128x128 pixel (Functional Post-Synthesis): & \SI{1721035100}{\ps}\\
Simulazione RESET WZs (Functional Post-Synthesis): & \SI{2754167600}{\ps}\\
\end{tabular}

\bigskip
Il risultato esprime una variazione del 36\% dal caso senza reset al caso con reset, evidenziando come l'architettura, per scelta progettuale,
è molto più efficiente quando non deve fare frequentemente la fase di setup, ossia la fase di caricamento delle WZ.

%----------------------------------------------------------------------------------------
%	SECTION 4
%----------------------------------------------------------------------------------------

\section{Conclusioni}

Si ritiene che l'architettura progettata rispetti anzitutto le specifiche, fatto che è stato verificato mediante estensivo testing sia casuale, che con test benches
manualmente scritti. Oltre a ciò l'architettura è stata pensata sulla base di una FSM, evitando l'utilizzo di Latch che avrebbero potuto creare l'instaurarsi di cicli infiniti.

Dal punto di vista del design, si è scelto di utilizzare una unica FSM senza componenti esterni.

%----------------------------------------------------------------------------------------

\end{document}
